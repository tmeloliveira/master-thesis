\documentclass[msc]{ppgccufmg}

\usepackage[english]{babel}
\usepackage[utf8]{inputenc}
\usepackage[T1]{fontenc}
\usepackage{type1ec}
\usepackage{graphicx}
\usepackage[a4paper,
  english,
  bookmarks=true,
  bookmarksnumbered=true,
  linktocpage,
  colorlinks,
  citecolor=black,
  urlcolor=black,
  linkcolor=black,
  filecolor=black,
  ]{hyperref}
\usepackage[square]{natbib}

\begin{document}

\ppgccufmg{
    title={- TBD -},
    portuguesetitle={- TBD -},
    authorrev={Oliveira, Thiago Melo de},
    cutter={D1234p},
    cdu={519.6*82.10},
    university={Federal University of Minas Gerais},
    course={Computer Science},
    portugueseuniversity={Universidade Federal de Minas Gerais},
    portuguesecourse={Ciência da Computação},
    address={Belo Horizonte},
    date={2020-03},
    advisor={Nivio Ziviani},
    coadvisor={Adriano Veloso},
    abstract=[brazil]{Resumo}{resumo},
    abstract={Abstract}{abstract},
    dedication={dedicatoria},
    ack={agradecimentos},
    keywords={Insert Keywords Here},
    epigraphtext={Truth and lie are opposite things.}{Unknown},
}

\chapter{Introduction}

The International Association for the Study of Pain (IASP) defines pain as "an unpleasant sensory and emotional experience associated with actual or potential tissue damage, or described in terms of such damage." \cite{merskey1994classification} The definition accompanying notes also establishes that "the inability to communicate verbally does not negate the possibility that an individual is experiencing pain and is in need of appropriate pain-relieving treatment."

As infants are unable to self-report pain, its diagnosis is much harder when compared to adults. Thus, studies have used non-verbal responses like facial expressions, crying sounds, and movements, alongside physiological measurements for better assessment. These methods have been tested and found to be reliable indicators of pain. Several observational scales have been published and verified based on them.

In the case of fetuses, however, there is still some discussion on the feasibility of developing such scales. As we have even more restrict possibilities of pain assessment, diagnosis becomes even more difficult.

\section{Motivation}

With studies showing that fetuses beyond a certain age also experience pain, early identification of this discomfort can be valuable in many situations.

One good example is intrauterine surgery, which may be of significant benefit in the future development and survival of the fetus. Early correction (prior to birth) of eventual problems will likely increase the odds of a healthy baby. These procedures, however, are quite invasive to the fetus and could eventually cause harm. The assessment of pain during the intrauterine life of a fetus is, therefore, a task with the potential of bringing great improvements to fetus life quality.

Another important discussion in the field is related to abortion. In the United States, a 2016 law from the state of Utah determines that women seeking abortion 20 weeks or more into a pregnancy will first have to be given anesthesia or painkillers. This procedure is intended not for them, but for the fetus.

This law generated a lot of ethical debate on the topic, as the exact week when a fetus experience pain is not well defined, and at the same time, abortion is only legal until a certain week. So discussion arrises not only if the fetus can or can not experience pain, but also as it may be the case that fetuses can only experience pain after weeks in which abortion is no longer possible.

Evidence of the presence of pain in this scenario would be a great contribution to this delicate situation and may even aid the decision by the doctors and the mother.

For both scenarios, the current standard for assessing pain in infants and fetuses relies on caregivers’ observation of specific behaviors such as facial expression, but these observations are subject to bias and can be affected by several factors, such as identity, background, culture, and gender, which may lead to inconsistent assessment and treatment of pain. An impartial perspective during the pain assessment process could also bring a more realistic and deterministic view on the subject.

\section{Contributions}

As fetuses respond to stimuli like anesthesia and present facial expressions which are evidence of pain, the goal of this project is to automate the detection of pain, as to generate automatic evidence of pain or not. 

During anesthesia application, studies have shown that it is feasible to detect pain facial expressions on the fetus. Given that this process depends on an experienced observant caregiver, it is time-consuming and is subject to the observant's bias, it would be great if we could use computational help to find expressions of pain. 

Such a system would be helpful in the generation of evidence that a fetus does experience pain. As the fetus behaves differently trough the manifestation of facial expressions on the presence of pain or not. Our ability to recognize pain is the subject of study for a long time with some studies going back even to Charles Darwin.

In this work, we have developed a system that given an image collected from a 4D ultrasound machine, can detect if it is in the presence of pain our not. 

This system, if integrated into the ultrasound machine itself, would allow the monitoring of the efficacy of anesthetic procedures, much like it is done on adults. If the surgeon detects that after the first anesthesia the fetus still shows signs of pain, he will be able to make better decisions and decide if another one is necessary.

This work would also open the way to explore the evolution of pain-related facial responses during fetal development. From the 20th gestational week onwards, fetuses have brain structures capable of showing signs of pain. This model would, therefore, allow for continuous monitoring of pain across time.

In summary, our main contribuitions are:

\chapter{Related Work}
\chapter{Methodology}
\chapter{Experimental Results}
\chapter{Conclusions and Future Work}

% Aqui vem a parte da bibliografia: use o comando \ppgccbibliography indicando
% apenas o nome do arquivo .bib (sem a extensão).
\ppgccbibliography{bibfile}

\end{document}
