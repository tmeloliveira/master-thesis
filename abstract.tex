Prolonged exposure to pain circumstances can have many side-effects on the life of a fetus and cause negative developmental consequences. Thus, pain assessment and management is made necessary to identify these scenarios early on. Even though numerous pain scales exist to help assess pain in neonates, until recently, no such method existed for detecting pain in fetuses. Based on these scales, some research has been developed to automatically assess pain through the means of analyzing images with computational help. Still, no such work had been developed for fetuses as well.

In this scenario, we propose the use of deep convolutional neural networks to construct a learning model capable of automatically detecting the presence of pain in fetuses through the evaluation of their facial expressions in images collected from 4-D ultrasound machines. To do so, we have taken advantage of transfer learning to use a pre-trained network on the task of face detection. Our results demonstrate the effectiveness of applying such methods with fetal images, and it confirms that transfer learning from a similar task performed better than if made from a general-purpose dataset. Above all, we show that it is possible to develop a model for automatically detecting pain in fetuses.

\keywords{Machine Learning, Fetal Pain, Automatic Pain Assessment}
