\chapter{Data}

Fetal therapy is a promising field in pediatric medicine, and prenatal surgery has become an option for an increasing number of babies with congenital disabilities. Regardless of its popularity increase, it is still a relatively rare procedure as it affects only a small percentage of pregnancies and offers risks for both the mother and the unborn baby. The procedure is also highly sophisticated and thus requires a skilled team, assisted with advanced technological resources to perform such complex procedures. The particularities of this topic, make research in the field very challenging.

At the same time, as fetuses are protected against biological and social effects while in utero, the circumstances offer an excellent opportunity for investigation of fetal behavior free of influences from the outside world. The study of fetal pain is a great example, as it is still a topic of debate if human fetuses feel pain or not. The Fetal Pain Study Group from the University of São Paulo was formed with the purpose of helping answer these questions. In the following section, we describe one of their most recent studies, which consisted of the assessment of pain trough facial expressions in fetuses.

\section{Fetal Pain Study}

Trough the use of high definition 4-D ultrasound machines, it is possible to record and observe fetal responses to different stimuli, and by looking at their facial expressions and body movements, one could potentially assess visual pain responses during the intrauterine life. This was done by the same study group, which proved the feasibility of using a pain scale, initially developed for acute pain assessment in neonates, in fetuses \citep{bernardes2018feasibility}. 

Based on this hypothesis, the Fetal Pain Study Group conducted a novel study, which is the first attempt to assess specific pain-related facial patterns in human fetuses. They were able to evaluate facial expressions after an anesthetic injection was administered before an intrauterine surgical procedure, which was used as a model of acute pain. As control conditions, two other scenarios were used, resting, and responses after an acoustic stimulus of a horn, which is routinely used to assess fetal wellbeing.

For the first group, fetuses diagnosed with diaphragmatic hernia with an indication of intrauterine surgery (fetoscopic endoluminal tracheal occlusion) were assessed in their preoperative period. The exact moment of the anesthetic puncture was recorded to capture the reaction of the fetus and its manifestations of pain. Videos from this group had two parts in it, first a baseline period defined as the first 45 seconds before the anesthesia puncture and second the 45 seconds immediately after the puncture. For this group, specifically, a second ultrasound machine was placed in the clinical room and operated by a fetal medicine specialist to monitor the fetus’s face and its expressions. The spatial set-up of the room can be seen in Figure \ref{fig:ultrasound}.

\begin{figure}[h!tp]
    \centering
    \includegraphics[width=.40\textwidth]{imgs/chap03_ultrasound_setup.jpg}
    \caption{Operating room surgery and face recording set-up. (1) Position of the mother; (2) chief surgeon who performed the puncture; (3) assistant surgeon who obtained the 4-D images; (4) surgical technologist; (5) ultrasound machine used in surgery focusing the fetal trachea/thigh; (6) ultrasound machine used for fetal face recording; and (7) an external camera. Diagram extracted from \citep{bernardes2018feasibility}}
    \label{fig:ultrasound}
\end{figure}

In order to measure and quantify pain, they have refined the Neonatal Facial Coding System (NFCS) to be more suitable for the application on fetuses. As fetuses can display facial expressions unrelated to pain \citep{Reissland2011}, the scoring system should be capable of discriminating acute pain responses from those at rest and from other non-painful stimuli that also trigger facial expressions, like the vibroacoustic sound of a horn. After refinement, indicators unable to discriminate between painful stimuli, and the control groups were removed. Likewise, indicators that were undetectable from static images were also not considered. Additionally, one item deemed relevant for the research was added: neck deflection.

The final scale thus contained the following seven items: brown lowering, eyes squeezed shut, deepening of the nasolabial furrow, open lips, horizontal mouth stretch, vertical mouth stretch and the new item neck deflection. Each item is considered one point if present or zero if absent on a given screenshot, then the present items are summed to give an overall score, and the scale ranges from zero to seven. The study concluded that no fetus in the control groups had a score higher than four, and at the same time, in the acute pain group, no score was less than five. These results allowed researchers to determine that a ``pain cut-off'' exists in the new seven-item scoring system.

\section{Data Description}

The data collected by the study and its findings present a unique opportunity to make further experiments. To the best of our knowledge, no publically available dataset exists with images or videos of fetuses while in pain exposure. This fact alone highlights the novelty and innovative aspects of the afforementioned study and of our research.

A total of 13 films were recorded from a 4-D ultrasound machine of the model Voluson E8 by General Electric, being 5 from the acute pain group, 4 from resting conditions and 4 from the exposure to a vibroacoustic sound. All the fetuses were in the third trimester of gestation, with an average of 31.1 weeks and a 2.8 standard deviation.

All mothers gave written informed consent to participate in the study and to record the behavioral reactions of the fetuses. The study was also approved by the ethics review board of the University of São Paulo Faculty of Medicine Clinics Hospital, under protocol number 2.649.528.