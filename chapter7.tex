\chapter{Conclusions and Future Work}

\section{Conclusions}

The results of our study presented in this dissertation are promising as we believe they move us towards the ultimate goal of automatically detecting pain in fetuses. Our learning model was indeed able to discriminate images of fetuses while in acute pain exposure from those in control groups of rest and acoustic stimulus. If confirmed on a larger dataset, we believe this work has the potential to influence and improve the current practice of assessing fetal pain.

We also think our work can serve as good evidence to help answer the question of when fetuses start to feel pain, as the exact gestational week in which they start showing pain responses is still not a consensus. By providing an unbiased and automatic approach for detecting pain, we could continuously monitor a fetus across many weeks, and an absence of pain may even be a good indicator of fetal wellbeing. However, a dataset with more variable gestational age would be necessary to train such a model.

On the other hand, if we do detect pain, the question arises as to what is the cause of it, as some condition may be present since our model accused the presence of pain. Is this condition some malformation? Is it a disease? Or is it related to chronic pain? It does open a range of possibilities but also brings awareness to future problems, which in some cases could be corrected with in-uterus repairs, with many benefits for the fetus.

As for the control group of acoustic stimuli, our model found it more difficult to discriminate it from pain, which was an expected outcome.  Because the group shares some common indicators with those of pain, it makes it indeed harder to predict, as shown in the original study that collected the data \citep{bernardes2018feasibility}. Nonetheless, we still believe that from a precautionary approach, it may be a good practice to investigate these cases, as they can be signs of discomfort or stress, and may also be caused by some conditions.

If our model eventually gets integrated into an ultrasound machine, it would make pain detection much simpler and easier to use, allowing continuous monitoring. This would be beneficial in many situations, especially during fetal surgery procedures, as it could aid anesthesiologists to see how effective their anesthesia was and aid the doctors while they perform the delicate procedures. Likewise, during routine prenatal ultrasound exams, this system could be the first to indicate pain or discomfort, which would then be further investigated.

It is also important to highlight that even though nurses and caregivers must assess the videos used as inputs by our model, we think a model trained on a range of different videos from different sources, would tend to be much more unbiased than an assessment of a single caregiver. This result is also a great benefit, as we can produce a system ideally free from bias factors such as identity, background, culture, and gender, which may lead to inconsistent assessment and treatment of pain.

\section{Future Work}

Our studies have shown that it is viable to construct a model capable of identifying the presence of pain on images of fetuses from 4-D ultrasound machines. A larger dataset is already being collected by the same fetal pain study group, which has the potential to confirm our results and produce models that are even more robust and accurate. We are also currently not able to explain why the model made such predictions or what is the main factors it considered for detecting pain. Thus, we identify as future work the following possibilities:

\begin{itemize}
    \item Evaluate our models on larger datasets. Even though the data is quite complicated to collect and studies with fetuses and infants usually have a small number of subjects, we think it would be very beneficial to experiment with our methodology in a more extensive number of fetuses. This addition could bring more variability into the model inputs in terms of fetal positions, gestational age, gender, image quality, and many other factors, which will end up producing a better model.
    
    \item Expand our system to include chronic pain. Monitoring the same fetuses at different gestational ages has the potential to identify the presence of chronic factors. A model that evaluates not only acute pain but also chronic pain could, therefore, help in this scenario, as it may lead to further investigation of what is causing the chronic pain and maybe be the first indicator that an intervention may be necessary.

    \item Include other types of features. As it is the case with pain scales, the combination of indicators is what tends to work best. Thus one could construct a model that takes as inputs not only images of the face and facial expressions but also other indicators such as sounds, body movements, physiological indicators, and biological markers. These new factors could help produce more robust models, and maybe identify conditions not visible through facial expressions only.
    
    \item Produce explainable models. Given a single fetus where the presence of pain has been identified, one should be able to visualize what are the most relevant features that the model analyzed to output its prediction. By making the decision of the models more transparent, one could point to the exact locations where the pain was present, which will lead to a better understating by fetal pain specialists. In an ideal scenario, these visual explanations should match the individual indicators present on the pain scale.

\end{itemize}

In summary, our main interests as future works are to help medical experts to understand the output of the models better and be more effective on pain assessment and management, which will eventually lead into improving overall fetuses life quality and well-being.
