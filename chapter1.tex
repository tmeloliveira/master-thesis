\chapter{Introduction}

The International Association for the Study of Pain (IASP) defines pain as ``an unpleasant sensory and emotional experience associated with actual or potential tissue damage, or described in terms of such damage'' \citep{merskey1994classification}. The definition accompanying notes also establishes that ``the inability to communicate verbally does not negate the possibility that an individual is experiencing pain and is in need of appropriate pain-relieving treatment.''

As newborns are unable to self-report pain, its diagnosis is much harder when compared to adults. Thus, specialists have used non-verbal responses like facial expressions, crying sounds, and movements, alongside physiological measurements for better pain assessment. These methods have been tested and found to be reliable indicators of pain. Several observational scales have been published and verified based on them, such as the Neonatal Infant Pain Scale (NIPS) \citep{Lawrence1993}, and the Neonatal Facial Coding System (NFCS) \citep{Grunau1998}.

In the case of fetuses we have even more restrict methods of pain assessment. Fortunately, some recent studies have shown the feasibility of applying these pain scales on a fetus through the use of 4-D (four-dimensional) ultrasound images \citep{bernardes2018feasibility}. This process allowed the monitoring of facial expressions on a fetus while they were exposed to noxious stimuli like an anesthetic puncture.

With recent advances in Artificial Intelligence (AI), the capacity of machines to detect patterns in images has largely improved, which consequently allowed for its application in diverse scenarios. Hence, these techniques could also be useful for pain assessment by helping matching patterns of facial expressions that are common indicators of pain.

\section{Motivation}

With studies showing that fetuses beyond a certain age can also experience pain \citep{Derbyshire2006, Derbyshire2020}, early identification of this discomfort can be valuable in many situations. 

One example is intrauterine surgery, which may be of significant benefit in the future development and survival of the fetus. Early correction, prior to birth, of congenital problems, will likely increase the odds of a healthy baby. These procedures, however, are quite invasive to the fetus and could eventually cause harm. The assessment of pain during the intrauterine life of a fetus is, therefore, a task with the potential of bringing significant improvements to fetus life quality. 

Another critical topic is abortion. In the United States, a 2016 law from the state of Utah determines that women seeking abortion 20 weeks or more into pregnancy will first have to be given anesthesia or painkillers \citep{healy2016nytimes}. This procedure is intended not for them but the fetus. 

This topic involves much ethical debate, as the exact week when a fetus starts experiencing pain is not well defined, and at the same time, abortion is only legal until a particular week \citep{Derbyshire2006}. So discussion arises not only if the fetus can or can not experience pain, but also as it may be the case that fetuses can only experience pain after weeks in which abortion is no longer possible. Evidence on the presence of pain in this scenario would be a significant contribution in such a delicate situation and may assist the decision by the doctors and the mother.

For both scenarios, the current standard for assessing pain in infants and fetuses relies on caregivers' observation of specific behaviors such as facial expressions. However, these observations are subject to bias and can be affected by several factors, such as identity, background, culture, and gender, which may lead to inconsistent assessment and treatment of pain. An impartial perspective during the pain assessment process could bring a more realistic and deterministic view on the subject. Hence, computational help would be of great use in finding evidence of pain and in effectively managing it.

\section{Thesis Statement}

We developed a learning model capable of automatically detecting the presence of pain in fetuses through the evaluation of their facial expressions in images collected from 4-D ultrasound machines. We have used modern deep learning techniques like transfer learning and data augmentation to find the best model. Our results demonstrated the effectiveness of applying such methods to the assessment of pain in fetuses and, to the best of our knowledge, is the first work attempting to do so.

\section{Contributions}

As fetuses have been shown to respond to stimuli like anesthesia with facial expressions indicating pain, the goal of this work is to help automate the pain assessment process and to generate unbiased evidence of pain. We have developed a process capable of detecting the presence of pain from images collected from 4-D ultrasound machines.

If this system is eventually integrated into the ultrasound machine itself, it will bring many benefits, such as the monitoring of anesthetic procedures efficacy, much like what it is done in adults. As an example, if the surgeon detects that after the first anesthesia, the fetus still shows signs of pain, he will be able to make better decisions and apply another one if necessary.

This work also opens the way to explore the evolution of pain-related facial responses during fetal development. Considering that after the 20th gestational week, fetuses start to develop brain structures capable of showing signs of pain, this model would, therefore, allow for continuous monitoring of pain across time.

In summary, our main contributions are:

\begin{itemize}
    \item We have created a systematic procedure for collecting and processing images of a fetus from videos of 4-D ultrasound machines. This procedure is also capable of detecting their facial landmarks. From this procedure, we have created a labeled database consisting of 230 images of 15 fetuses with facial expressions while in the manifestation of pain and two other control conditions. To the best of our knowledge, no such database existed.
    
    \item We have developed a learning model capable of detecting the presence of pain indicators from images of a fetus's face. We believe this is good evidence towards an unbiased pain assessment process. This novel approach has the potential to improve the pain assessment process significantly on fetuses. It would facilitate pain management by the doctors and caregivers and could even be the first indicator of discomfort or distress, leading to earlier intervention if necessary.
    
    \item We have shown that transfer learning with a network pre-trained with the face recognition task transfers well to fetus images even though the domain is different. We have achieved an area under the ROC curve of 0.69 on the classification task between pain and non-pain.
\end{itemize}

\section{Organization}

The rest of this dissertation is structured as follows. First, Chapter 2 discusses related work in pain assessment. Chapter 3 introduces some background concepts on deep learning, necessary to further understand our work. Chapter 4 describes the fetal pain assessment study, and also introduces our dataset. Chapter 5 follows with our methodology, including our learning model. Then, Chapter 6 describes our validation process, as well as our experimental results. Chapter 7 concludes the dissertation and present future work possibilities.
