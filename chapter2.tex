\chapter{Related Work}

In this chapter we present the most relevant research results that guided our work, exposing the methodologies used by the authors and how they correlate to ours. Research on automatic pain assessment has a great intersection between the medical field and applied machine learning both for fetuses and infants \citep{ZamzmiPGKSA16, Bellieni2012}. Thus, we start by exploring the causes of pain in infants and fetuses, its main indicators, the available pain scales and methods to assess it. 

Next, we present an overview of deep learning methods that form the basis of our methodology. We also explain how our work is different from the mentioned ones since we explore automatic pain assessment from a novel aspect with fetuses.

\section{Pain}

Pain is a universal form of distress present in humans and some other animals. Acute and chronic pain are very common in the population and constitute widespread public health problems. Its prolonged presence could cause many adverse consequences, including psychological effects. This is especially true in the case of fetuses and infants.

The study of neonatal pain seems to have begun as early as the 1870s, when Dr. Flechsig proposed it was unlikely that neonates could feel pain because their neuronal myelination was not complete \citep{cope1998neonatal}. Charles Darwin's book written a couple years later agreed with this view, as he wrote that "an infants pain expressions were related to reflexes only" \citep{darwin1872expression}. Even as late as the 1950s some pediatric surgeries were performed without anesthesia and analgesia \citep{cope1998neonatal}.

% -------------------------------------------------------------
% FAZER CONEXÃO COM O SURGIMENTO DO ESTUDO DE DOR EM FETOS AQUI
% -------------------------------------------------------------

\subsection{Pain Indicators}

Fetuses and infants can produce different signals of pain which can be decoded to both identify its presence and to measure its level. These signals come from a variety of sources such as facial expressions, crying sounds, body movements, physiological indicators, and biological markers \citep{Bellieni2012}. 

Even though we have this many indicators, pain identification is a challenging task as we have the manifestation of the same indicators present in similiar feelings such as anger, hunger or stress. To address this issue, the recommendation is that these indicators should be used in combination with each other, because most of the time their presence alone is not sufficient. 

Crying, for instance, can also be generated by hunger or anger, therefore it can not be used as a sole indicator of pain. In the past, it was used to believe that different emotions resulted in different types of crying, but this theory has been refuted as it was later discovered that what causes the difference is not the cause of distress, but rather its intensity. 

Some pain scales which combine features of crying and other indicators do exist, and a fetus can show a homolog of crying \citep{Gingras2005}. So this could further be explored. \cite{abs-1909-02543}

% -------------------------------------------------------------
% VERIFICAR O PAPER DE CRYING SOUNDS E SUAS REFERÊNCIAS
% -------------------------------------------------------------

Physiological indicators have the limitation that they are subject to variations due to underlying illness \citep{sweet1998physiological}.

Body movements have also been pointed out to be indicators of pain, as even fetuses already present whitdrawn reflexes during stressfull procedures \citep{Zimmermann1991}. On the other hand, they can also be misleading as the movements may be caused by  other factors. 

Many studies have shown that biological markers like stress hormones (cortisol, adrenaline, and beta-endorphins) are increased in concentration in the blood in the presence of pain \citep{giannakoulopoulos1994fetal}. But the problem of these indicators is they depend on results from laboratory tests, which makes it unfeasible to use during clinal trials.

% -------------------------------------------------------------
% FALAR DE FACIAL EXPRESSIONS COMO INDICADORES
% -------------------------------------------------------------

\subsection{Pain Scales}

Various neonatal pain scales were developed using the many indicators mentioned in the previous section. These multidimensional scales are used by caregivers to assess pain with behavioral and physiological indicators. The most common scales are:

\begin{itemize}
    \item Neonatal Infant Pain Scale (NIPS)
    \item Face, Legs, Activity, Crying and Consolability (FLACC)
    \item Neonatal Facial Coding System (NFCS)
\end{itemize}

These scales were all developed with neonates in mind, as some indicators observations are not easily mesured in fetus. But recent studies by \citeauthor{bernardes2018feasibility} have reported that the use of the NFCS is feasible to detect pain-related facial expressions compared with the rest condition in a randomized and blinded assessment report.

But the scales also have some limitations, as they are highly dependent on observant bias, require specific training for proper utilization and are not able to monitor pain in a continuous manner. Thus developing tools that are capable of doing this job automatically and contininuously is highly compelling as they can generate a more consistent pain assessment.

\section{Automatic Pain Assessment}

The first work attempting to automatically assess pain in newborns have emerged in 2006  with the development of the iCOPE database. At the time neural networks were not as popular and advanced as they are today. The first attempts made use of traditional techniques such as PCA, LDA, and SVM. Still, the experimental process they have developed is somewhat similar to what is used today. 

Their database consisted of 216 images from 26 infants being 13 girls and 13 boys. The images were collected in 5 different conditions and later divided into groups for classification non-pain

% , which consisted of [...] and 

Their first studies considered the best scenario where the system would be able to train on a fetus and evaluate that same fetus later on. But as permanence in the baby nursery is quite short, they also had to consider the case where this was not possible. Thus the validation process had to consider this other scenario where the classifier had to be trained beforehand and evaluated on images of a new baby which was not in the dataset distribution previously.

Given the number of subjects is small, it was feasible to use a leave one (subject) out validation process. Which consisted of using 25 for training and 1 for test and so forth.

Our work as adopted a similar approach of not having only images of rest and pain, but we've also added a third set of images from other stimuli of a horn, which had the intention of causing discomfort, but no pain. 

\section{Transfer Learning for Pain Assessment}

Transfer learning is often used for domain adaptation. The networks had the capability of learning fundamental basic features from the data such as corners, edges in the first layers, they then learn other things in the later layers. But overall, with fine-tuning the domain adaptation is possible.