\chapter{Conclusions and Future Work}

\section{Conclusions}

The results of our study presented in this dissertation are promising as we believe they move us towards the ultimate goal of automatically detecting pain in fetuses. Our learning model was indeed able to discriminate images of fetuses while in acute pain exposure from those in control groups of rest and acoustic stimulus. If confirmed on a larger dataset, we believe this work has the potential to influence and improve the current practice of assessing fetal pain.

We also think our work can serve as good evidence to help answer the question of when fetuses start to feel pain, as the exact gestational week in which they start showing pain responses is still not a consensus. By providing an unbiased and automatic approach for detecting pain, we could continuously monitor a fetus across many weeks, and an absence of pain may even be a good indicator of fetal wellbeing. However, a dataset with more variable gestational age would be necessary to train such a model.

On the other hand, if we do detect pain, the question arises as to what is the cause of it, as some condition may be present since our model accused the presence of pain. Is this condition some malformation? Is it a disease? Or is it related to chronic pain? It does open a range of possibilities but also brings awareness to future problems, which in some cases could be corrected with in-uterus repairs, with many benefits for the fetus.

As for the control group of acoustic stimuli, our model found it more difficult to discriminate it from pain, which was an expected outcome.  Because the group shares some common indicators with those of pain, it makes it indeed harder to predict, as shown in the original study that collected the data \citep{bernardes2018feasibility}. Nonetheless, we still believe that from a precautionary approach, it may be a good practice to investigate these cases, as they can be signs of discomfort or stress, and may also be caused by some conditions.

If our model eventually gets integrated into an ultrasound machine, it would make pain detection much simpler and easier to use, allowing continuous monitoring. This would be beneficial in many situations, especially during fetal surgery procedures, as it could aid anesthesiologists to see how effective their anesthesia was and aid the doctors while they perform the delicate procedures. Likewise, during routine prenatal ultrasound exams, this system could be the first to indicate pain or discomfort, which would then be further investigated.

It is also important to highlight that even though nurses and caregivers must assess the videos used as inputs by our model, we think a model trained on a range of different videos from different sources, would tend to be much more unbiased than an assessment of a single caregiver. This result is also a great benefit, as we can produce a system ideally free from bias factors such as identity, background, culture, and gender, which may lead to inconsistent assessment and treatment of pain.

\section{Future Work}